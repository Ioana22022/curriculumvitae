\documentclass[10pt,a4paper]{moderncv}

\moderncvtheme[blue]{classic} 
\usepackage[utf8]{inputenc}   
\usepackage{anyfontsize}
%\usepackage[scale=0.9]{geometry}
\usepackage[top=0.5cm, bottom=0.5cm, left=0.5cm, right=0.5cm]{geometry}
\usepackage{graphicx}

\firstname{Ioana-Laura}
\familyname{Popescu\\}
\address{Bucharest, Romania}   
\mobile{+40756045984}              
\email{ioanalaura10@gmail.com}
\makeatletter
%\renewcommand*{\bibliographyitemlabel}{\@biblabel{\arabic{enumiv}}}
\makeatother

%\usepackage{multibib}
%\newcites{book,misc}{{Books},{Others}}
\nopagenumbers{}                         
\begin{document}
\maketitle
\fontsize{9}{11}
\selectfont
\section{Work Experience}
\cventry{Sep, 2019 – Present}{Software Engineer - System Health Team}{Harman}{
	As part of System Health Team, I continued on the same project built for an embedded Linux platform, and first started on the Performance Domain. Having the code being mostly written so the number of tickets regarding bugs and poor performance was visibly growing, most of them due to lack of best practices used by the inital developers. Several of my tasks included analysis of said ticket/documenting on the problem, finding the root cause, provide solution, eventually an implementation (if the problem was in SH area). An example was improving the way another application from SH was fetching information from the systemd-journal with sd\_journal API. CPU usage was improved by ~30\%. Following this, my responsabilities were expanded and got implied on several more features on the mentioned application, and also on a third one, which has been requested by the architects to be used as a decoder tool to ease the way some typical bugs were analysed. Continuing to provide support on most tickets arriving in SH team, as having expertise in four different domains for the current project.
Languages: C/C++, Bash, Java (when explicitly needed)	
Tools used: GDB, lttng/Eclipse Trace Compass, Yocto, DLT-Viewer, GitHub, SVN, Jira, Crucible, Atlassian Confluence}{}{}
\cventry{Sep, 2018 – Aug, 2019}{Software Engineer - System Lifecycle Team}{Harman}{
	System Lifecycle Team was the application which was responsible with the boot-up and shutdown of the system. When arrived, the number of bugs reported regarding this domain was exponentially growing, most of the activities included bug fixing. Developed features for the Engineering Menu on the infotainment system. One of them has been requested by the client with one week deadline, to fix wrongly displayed version and number on the System Information. Another implied rearranging the systemd dependencies configuration of the entire system so the boot time get's boosted. Boot time has been increased by ~41\%.  
}{}{}
\cventry{Oct, 2017 – March, 2018}{Concept Engineering}{Infineon Technologies, Romania}{ As part of the concept engineering team, main responsabilities included development of python scripts for processing data regarding pressure and magnetic sensors models. Languages: Python/Pandas module, Bash. Tools: PyCharm, Matlab/Simulink}{}{}
\cventry{June, 2017 – September, 2017}{Summer Internship at ADAS Department, Algorithm Development (C++)}{Continental Automotive, Romania}{Contributed to implementation of the Pothole and Speedbump Detector, worked on features/bug fixes such as: reducing the number false positive outcomes, keeping track of SB's detected}{}{}

\section{Education}
\cventry{2018 - 2020}{Universitatea Politehnica București, Facultatea de Automatică și Calculatoare}{Master of Engineering}{Advanced Computer Architectures}{}{}
\cventry{2014 - 2018}{Universitatea Politehnica București, Facultatea de Automatică și Calculatoare}{Bachelor's degree}{}{}{}
\cventry{2010 - 2014}{Colegiul Național "Mircea cel Batr\^{i}n" Constan\c{t}a}{Matematică - informatică}{}{}{}{}

\section{Projects}

\cvline{MPTCP-aware HTTP Load Balancer}{My task was to implement an \textbf{MPTCP}-aware \textbf{HTTP} load-balancer using \textbf{OpenFlow}. The experimental setup consisted of installing an mptcp-capable kernel, configuring path-manager and install a \textbf{Mininet} VM to simulate a network consisting of three hosts and an OpenFlow switch. When the client initiated connections, the controller hashed the IP addresses and ports and use the hash to pick one of the two servers. The controller also examines the SYN+ACK messages and learn the server's key in order to create the servers token, so when the client creates the new subflow, it examines the token and send the subflow to the appropiate server.}
\cvline{Embedded System based on Versatile Platform Baseboard}{Built using \textbf{Yocto} and emulated with \textbf{QEMU}. Had a basic image consisting of : kernel Image already compiled, layer with rpi basic image recipe and other layers. The system consisted of the following features: able to send and receive data over predefined format, had user root with defined password, configured it’s own IP using \textbf{DHCP}, able to respond on \textbf{SSH} on a configured port to <some name>.local Avahi/DNS daemon, could communicate through tty/AMA0 serial interface (this was configured on the QEMU command), had a Web interface where it could plot the history data and also send data through it (plotly.js). Receiving and sending data were developed with python scripts, that were later added on the system’s 2nd  run level through an additional Yocto recipe.
}
\cvline{Designing CPU (8086) in Verilog}{One of the most challenging coursework was to design a basic CPU that could support instructions such as ADD, SUB, MOV, POP, PUSH, JMP, AND, OR, XOR, conditioned JMP and others. Sketch code was provided from laboratory.}
\cvline{Million Song Dataset}{Inspected a selected subset of the Million Song DataSet of 10000 songs using \textbf{Spark} framework and implemented three tasks: Exposing the loudness war trend, with a map-reduce algorithm and plotted the evolution using \textbf{Gnuplot}. Quantified the tendencies for crossover between genres. Result was a crossover matrix, where cell(x, y) denoted how much of a crossover there is between genre x and genre y.}{}{}{}{}

\section{Personal skills}
\cvline{Languages}{Română (limbă maternă), Engleză (fluent level). }
\cvline{Computer science}{\textbf{Programming Languages}: C, C++, Assembly (NASM syntax), Verilog (basic);  \textbf{Scripting}: Bash, Python; \textbf{Environment}: Linux, QNX; \textbf{Debugging Tools}: GDB, Valgrind, DLT; \textbf{Other}: Xen hypervisor, GFS, Hadoop, Spark, CDN's; \textbf{Only if necessary}: HTML, CSS, JavaScript, PHP(basic), Angular JS(basic), LateX}

\section{Contests and Awards}
\cventry{2013}{Bronze Medal}{Olimpiada Națională de Matematică, Etapa National\u{a}}{Brasov}{}{}

\end{document}
